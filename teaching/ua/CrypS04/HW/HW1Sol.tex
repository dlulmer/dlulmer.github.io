\nopagenumbers
\def\ex#1{\item{\bf#1}}
\def\ct#1{{\it #1\/}}

\centerline{Math 445}
\centerline{Introduction to Cryptography}
\centerline{Homework Solutions}
\centerline{January 26, 2004}
\bigskip

\ex{2.13.2}  Using the standard encoding of a-z as 0-25, \ct{howareyou} becomes

{\narrower 8 14 22 0 17 4 24 14 20

}
Applying the affine transformation $x\mapsto 5x+7\pmod{26}$ yields:

{\narrower\obeylines $7\mapsto42\equiv16\pmod{26}$
$14\mapsto77\equiv25\pmod{26}$
$22\mapsto117\equiv13\pmod{26}$
$0\mapsto7\pmod{26}$
$17\mapsto92\equiv14\pmod{26}$
$4\mapsto27\equiv1\pmod{26}$
$24\mapsto127\equiv23\pmod{26}$
$14\mapsto77\equiv25\pmod{26}$
$20\mapsto107\equiv3\pmod{26}$

}
Translating back to letters gives the ciphertext: \ct{QZNHOBXZD}.  
The decryption function is $y\mapsto a'x+b'$ where $5a'\equiv1\pmod{26}$ and $b'=-7a'\pmod{26}$.  Brute force (or more sophisticated methods ....) reveals that $a'=21$ and then $b'=9$.  Straightforward calculation shows that it works, i.e., it transforms the ciphertext back into the plaintext.


\bigskip

\ex{2.13.4} Since  \ct{C} (2) decrypts to \ct{h} (7) and \ct{R} (17) decrypts to \ct{a} (0), the decryption function $y\mapsto a'y+b'$ satisfies  the equations
$$\eqalign{7&\equiv 2a'+b'\pmod{26}\cr0&\equiv 17a'+b'\pmod{26}.}$$
The second equation gives $b'\equiv-17a'\equiv9a'\pmod{26}$.  Substituting into the first equation then implies that $7\equiv11a'\pmod{26}$.  Since $19\cdot11\equiv1\pmod{26}$, we have $a'\equiv19\cdot7\equiv3\pmod{26}$ and $b'\equiv9\cdot3\equiv1\pmod{26}$.  Thus the decryption function is $y\mapsto 3y+1\pmod{26}$ and the message decrypts to \ct{happy}.

\bigskip

\ex{2.13.5}  There is no advantage because the composition of two affine ciphers is another affine cipher.  Indeed, if we compose $y=ax+b$ with $z=cy+d$, we get
$$z=cy+d=c(ax+b)+d=(ac)x+(bc+d).$$
which is just an affince cipher with key $(ac, bc+d)$.

\bigskip

\ex{2.13.6} If we work modulo 27, then the legitimate keys are $(a,b)$ where the greatest common divisor (gcd) of $a$ and 27 is 1 and $b$ is arbitrary.  Since $27=3^3$, $\gcd(a,27)=1$ if and only if 3 does not divide $a$.  Moreover, if $a'\equiv a\pmod{27}$ and $b'\equiv b\pmod{27}$ then $(a,b)$ and $(a',b')$ give the same encryption function.  In other words, we should regard $a$ and $b$ as numbers modulo 27.  So we get every key exactly once if we choose $a$ from the set $\{1,2,4,5,7,8,\dots,25,26\}$ and $b$ from the set $\{0,1,2,\dots,25,26\}$.  There are 18 choices for $a$ and $27$ choices for $b$, so $486$ keys in all. \hfil\break
Working modulo 29 the story is similar, except that 29 is prime, so $\gcd(a,29)=1$ for any $a$ not divisible by 29.  Thus we have 28 choices for $a$ and 29 choices for $b$ and so 812 keys in all.

\bigskip

\ex{2.13.7} Suppose that $\gcd(\alpha,26)=d>1$.  Then $d$ divides 26 and so $(26/d)$ is an integer.  Let $x_2$ be any integer modulo 26 and set $x_1=x_2+(26/d)$.  Since $d>1$, we have $0<(26/d)<26$ and so $x_1\not\equiv x_2\pmod{26}$.  Now we calculate the encryption of $x_1$:
$$\eqalign{\alpha x_1+\beta&=\alpha(x_2+(26/d))+\beta\cr
&=\alpha x_2 +\beta+\alpha(26/d).}$$
But $\alpha(26/d)=(\alpha/d)26$ and since $d$ divides $\alpha$, the quantity $(\alpha/d)26$ is an integer times 26.  Thus the calculation above shows that
$$\alpha x_1+\beta\equiv \alpha x_2+\beta\pmod{26}.$$
This means two different plaintext characters (namely $x_1$ and $x_2$) encrypt to the same ciphertext character and so we will not be able to decrypt.

\bigskip

\ex{2.14.2}  Using Mathematica:
\medskip
{\narrower\obeylines\tt
{\rm The ciphertext is stored in {\tt lcll}:}

In[105] := lcll
Out[105] = lcllewljazlnnzmvyiylhrmhza

\medskip
{\rm Do a frequency count:}

In[106] := frequency[lcll]
Out[106] =
$$\{\{a,2\},\{b,0\},\{c,1\},\{d,0\},\{e,1\},\{f,0\},\{g,0\},\{h,2\},\{i,1\},$$
$$\{j,1\},\{k,0\},\{l,6\},\{m,2\},\{n,2\},\{o,0\},\{p,0\},\{q,0\},\{r,1\},$$
$$\{s,0\},\{t,0\},\{u,0\},\{v,1\},\{w,1\},\{x,0\},\{y,2\},\{z,3\}\}$$

\medskip
{\rm The most common letter is l so we guess this is a shift by 7.  Try it out: }

In[107] := affinecrypt[lcll,1,-7]
Out[107] = eveexpectseggsforbreakfast

}

\bigskip

\ex{2.14.3} This is like problem 13.4: we need to solve
$$\eqalign{8&\equiv 4a'+b'\pmod{26}\cr5&\equiv3a'+b'\pmod{26}}$$
Subtracting the second equation from the first gives
$$3\equiv a'\pmod{26}$$
and substituting into either equation gives
$$22\equiv b'\pmod{26}.$$

Now use Mathematica to do the decryption:
\medskip
{\narrower\obeylines\tt
In[108] := edsg
Out[108] = edsgickxhuklzveqzvkxwkzukcvuh

\medskip
In[109] := affinecrypt[edsg,3,22]
Out[109] = ifyoucanreadthisthankateacher

}

\end

