\documentclass{article}
%\setlength{\topmargin}{-1in}
\setlength{\textheight}{9in}
\setlength{\evensidemargin}{-.25in}
\setlength{\oddsidemargin}{-.25in}
\setlength{\textwidth}{6.5in}

\begin{document}
\title{Math 160/263 Minitab Assignment \# 8 - Unix Version} 
\author{Chapter 4 - Sampling Distributions}
\date{}
\maketitle

\begin {enumerate}

\item  A snack-food company uses a machine to package bags of pretzels.  
The bags are supposed to contain 454 grams (g).  In fact, the contents 
vary according to a normal distribution with mean $\mu=454$ g and standard 
deviation $7.5$ g.

\begin{enumerate}

\item  Use the {\bf CDF} command with the {\bf NORMal} subcommand to find 
the probability that an individual bag contains less than 445 g.

\item  Use the {\bf LET} command to find the mean and standard deviation 
for the contents of the bags in a carton of eight.  Use the {\bf CDF} command 
to find the probability that the mean contents of the bags in a carton of 
eight is less than 445 g.

\end{enumerate}


\item  A survey by the National Fisheries Institute showed that 25\% of 
adults comsume seafood two or three times a week.
If 100 adults are selected at random, then the count of respondents 
that consume seafood two or three times a week 
has a binomial distribution with parameters $n=100$ and $p=0.25$.

\begin{enumerate}

\item  Use the commands given below to simulate 100 observations of this 
random variable in each of 40 columns of the worksheet.

\begin{verbatim}

     MTB > RANDom 100 c1-c40;
     SUBC> BINOmial 100 0.25.

\end{verbatim}

\item  Use the commands given below to compute the row means for the 
first 10 columns, the first 20 columns, and all 40 columns.

\begin{verbatim}

     MTB > RMEAN c1-c10 c41
     MTB > RMEAN c1-c20 c42
     MTB > RMEAN c1-c40 c43

\end{verbatim}

\item  Create graphical and numerical descriptions of the results in 
columns 1, 41, 42, and 43.

\item  Briefly describe how the shapes of the distributions compare 
with one another, how the means compare with one another, and 
how the standard deviations compare with one another.

\end{enumerate}


\end{enumerate}

\end{document}

