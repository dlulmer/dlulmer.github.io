\documentclass{article}
\setlength{\topmargin}{-.25in}
\setlength{\textheight}{12in}
\setlength{\evensidemargin}{-.25in}
\setlength{\oddsidemargin}{-.25in}
\setlength{\textwidth}{6.5in}

\begin{document}
\title{Math 160/263 Minitab Assignment \# 3 - Windows Version} 
\author{Chapter 2 - Correlation and Linear Regression}
\date{Worksheet Name - data3.MTW}
\maketitle

\begin{enumerate}

\item  The city and highway gas mileages for 29 midsize cars and
26 four-wheel-drive sport utility vehicles from the 1998 model
year are given in data3.MTW.

\begin{enumerate}

\item  Use the {\bf Graph} menu command to make a scatterplot of highway 
mileage against city mileage, using different symbols for the midsize 
cars and SUVs.

\item  Briefly describe the direction, form, and strength of the
relationship between city and highway mileage.

There is one clear outlier.  This is the Volkswagen Pasat with
disel engine.  Because this is the only vehicle with a diesel
engine, remove it from the data before doing any further analysis.

\item  Use the {\bf Stat \boldmath $>$ \unboldmath Basic Statistics} to 
find the correlation between city mileage and highway
mileage for the entire set of data, for the midsize cars alone, and
for the SUVs alone.

\item  Briefly explain why the three correlations are similar.

\end{enumerate}


\item  The average numbers of steps per second for a group of top female 
runners at different speeds are also given in data3.MTW.  The speeds are 
in feet per second.

\begin{enumerate}

\item  We would like to predict steps per second from running speed.  Use
the {\bf Graph} menu command to make a plot of the data with this goal in
mind.

\item  Describe the direction, form and strength of the relationship 
between speed and steps per second.

\item  Use the {\bf Stat $>$ Regression} menu command to find the
least-squares regression line of steps per second on
running speed, and draw this line on the scatterplot.

\item  Does running speed explain most of the variation in the number of
steps that a runner takes per second?  Find $r^2$ and use it to answer

\item  If we wanted to predict running speed from a runner's steps per
second, would we use the same line?  Explain.  Would $r^2$ stay the same?

\end{enumerate}


\end{enumerate}
\end{document}


