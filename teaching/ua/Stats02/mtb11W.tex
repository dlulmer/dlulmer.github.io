\documentclass{article}
%\setlength{\topmargin}{-1.5in}
\setlength{\textheight}{11in}
\setlength{\evensidemargin}{-.25in}
\setlength{\oddsidemargin}{-.25in}
\setlength{\textwidth}{6.5in}

\begin{document}
\title{Math 160/263 Minitab Assignment \# 11 - Windows Version} 
\author{Chapter 6 - Inference for Distributions}
\date{Worksheet Name - data11.MTW}
\maketitle

\begin {enumerate}

\item  The monthly fees (in dollars) paid by a random sample of 50
users of commercial Internet service providers in August 2000
are given in data11.MTW.

\begin{enumerate}

\item  State the appropriate $H_o$ and $H_a$ for a statistical test of the
claim that the mean cost for all Internet users differs from \$20
per month.  Be sure to identify the parameter appearing in the hypotheses.

\item  Make a graphical check for outliers or strong skewness in the 
data that you will use in your statistical test, and report your 
conclusions on the validity of the test.

\item  Use the {\bf Stat $>$ Basic Statistics $>$ 1-Sample t} menu command
to carry out the test.  Can you reject $H_o$ at the 5\% 
significance level?  At the 1\% significance level?

\item  Give a 90\% confidence interval for the mean monthly cost for
all Internet users.

\end{enumerate}


\item  A study compared various characteristics of 68 healthy and
33 failed firms.  One of the variables was the ratio of current assets
to current liabilities.  Roughly speaking, this is the amount that the
firm is worth divided by what it owes.  The data are given in data11.MTW.

\begin{enumerate}

\item  Describe the data graphically.  Are there outliers or strong
skewness that might prevent the use of $t$ procedures?

\item  State the hypotheses for a statistical test of the claim that
failed firms have a lower ratio of current assets to current liabilities.

\item  Carry out the test using the
{\bf Stat $>$ Basic Statistics $>$ 2-Sample t} menu command.  Is the
result significant at the 10\% level.  At the 5\% level?  At the 1\% level?

\item  Give a 95\% confidence interval for the difference between
the mean ratio of current assets to current liabilities for healthy
firms and the mean ratio of current assets to current liabilities for
failed firms.

\end{enumerate}


\end{enumerate}
\end{document}

