\documentclass{article}
%\setlength{\topmargin}{-.75in}
\setlength{\textheight}{11in}
\setlength{\evensidemargin}{-.25in}
\setlength{\oddsidemargin}{-.25in}
\setlength{\textwidth}{6.5in}

\begin{document}
\title{Math 160/263 Minitab Assignment \# 10 - Unix Version} 
\author{Chapter 5 - Introduction to Inference}
\date{Worksheet Name - data10.MTW}
\maketitle

\begin {enumerate}

\item  A brewery filling machine is adjusted to fill quart bottles 
with a mean of 32.0 ounces of ale and a standard deviation of 0.05 
ounce.  In order to determine whether the machine is working properly, 
a quality-control engineer measured the contents of 20 randomly selected 
bottles.  The results are given in data10.MTW.

\begin{enumerate}

\item  Use the {\bf ZINTerval} command to find a 90\% confidence interval 
for the mean contents of all bottles of ale filled by the machine.

\item  Is there significant evidence at the 10\% level that the mean 
contents of the bottles is not 32.0 ounces?  State hypotheses and base a 
test on the confidence interval from (a).

\end{enumerate}


\item  Sulpher compounds cause ``off-odors'' in wine, so oenologists 
(wine experts) have determined the odor threshold, the lowest 
concentration of a compound that the human nose can detect.  For 
example, the odor threshold for dimethyl sulfide (DMS) is given in the 
oenology literature as 25 micrograms per liter of wine ($\mu$g/l).  
Untrained noses may be less sensitive, however.  The DMS odor thresholds 
for 10 beginning students of oenology are given in data10.MTW.  Assume 
that the standard deviation of the odor threshold for untrained noses is 
known to be $\sigma=7 \mu$g/l.

Is there convincing evidence that the mean odor threshold for beginning 
students is higher than the published threshold, 25 $\mu$g/l?  Use the 
{\bf ZTESt} command to carry out the significance test. 


\item  The Stanford-Binet ``IQ test'' is adjusted so that
the scores for each age group of children are approximately normally
distributed with mean 100 and standard deviation 15.

\begin{enumerate}

\item  Use the {\bf RANDom} command with the {\bf NORMal} subcommand to
simulate 50 scores on the ``IQ test'' in each of 20 columns of the worksheet.

\item  Use the {\bf ZTESt} command to carry out a test of the null 
hypothesis that $\mu=100$ for each of the 20 samples of scores on the 
``IQ test''.  How many times is the null hypothesis rejected at the 5\% 
level?  How many times is the null hypothesis rejected at the 1\% level?

\end{enumerate}


\end{enumerate}

\end{document}

