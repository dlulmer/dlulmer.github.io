\documentclass{article}
\setlength{\topmargin}{-.25in}
%\setlength{\textheight}{9in}
\setlength{\evensidemargin}{-.25in}
\setlength{\oddsidemargin}{-.25in}
\setlength{\textwidth}{6.5in}

\begin{document}
\title{Math 160/263 Minitab Assignment \#1 - Unix Version} 
\author{Chapter 0 - Introduction to Minitab}
\date{Worksheet Name - data1.MTW}
\maketitle

\noindent
{\em  The purpose of this assignment is to introduce you to the 
statistical software package, Minitab.  You will learn how to enter, edit, 
and view data as well as open and save files.}

\begin{enumerate}

\item  To start Minitab, type {\tt minitab} at the $>$ prompt.  When you 
see the MTB $>$ prompt, you are in Minitab.  
At this prompt type,
\begin{center}
{\bf OUTFile} '{\em filename}'
\end{center}
to open an {\em outfile}.  (Note that the filename must be in single 
quotes and should not contain any periods or other special characters, 
spaces, or extensions.)

\vspace{.1in}

There are two critical commands in Minitab.  Failure to use either one of
these commands may unfortunately result in the loss of your work.  The
{\bf OUTFile} command is the first ``critical'' command.

\item  Use the {\bf RETRieve} command to load the {\em worksheet} 
that contains the data for this assignment.  NOTE:  This and all future 
worksheets will be in /xdisk/mendel.  This path name must be entered before  
the name of the worksheet.  For example, you would use 
/xdisk/mendel/data1.MTW to specify the worksheet for this assignment.

\item  Use the {\bf INFOrmation} command to list the contents of the 
{\em worksheet} and the {\bf PRINt} command to display the data.

\item  The data for the textbook used in Math 160 was omitted from the 
{\em worksheet}.  Use the {\bf INSErt} command to append columns 1 through 
3 with the data given below.

\begin{center}
\begin{tabular}{cll}

Textbook                              & Author & Price \\
\hline
The Basic Practice of Statistics, 2/E & Moore  & 77.75  

\end{tabular}
\end{center}

\item  Use the {\bf PRINt} command to display the revised data.

\pagebreak

In order to conduct a thorough analysis of the prices of these books, it 
would be desirable to know the length of each book.  This information is 
given below.

\begin{center}
\begin{tabular}{lc}

\multicolumn{1}{c}{Textbook}                       & Pages \\
\hline
A Brief Introduction to Probability and Statistics & 640 \\
Elementary Statistics, 5/E                         & 864 \\
Elementary Statistics, 8/E                         & 800 \\
Elementary Statistics: Picturing the World, 2/E    & 704 \\
Exploring Statistics, 2/E                          & 940 \\
First Course in Statistics, A, 8/E                 & 576 \\
Introduction to Statistics                         & 752 \\
Introduction to the Practice of Statistics, 3/E    & 828 \\
Introductory Statistics, 4/E                       & 880 \\
Introductory Statistics, 6/E                       & 984 \\
Learning Statistics with Real Data                 & 300 \\
Mind on Statistics                                 & 592 \\
Statistics, 9/E                                    & 880 \\
The Basic Practice of Statistics, 2/E              & 619

\end{tabular}
\end{center}

\item  Use the {\bf NAME} command to assign the name 'Pages' to column 4, and 
use the {\bf SET} command to input the number of pages for each book.

\item  Use the {\bf INFOrmation} and {\bf PRINt} commands to verify that 
the number of pages for each book has been added to the {\em worksheet}.

\item  Now suppose that you would like to compute the price per page.
Use the {\bf HELP} command to obtain instructions for 
the use of the {\bf LET} command, the use the {\bf LET} command to do 
the desired computation.  (Store the results in column 5.)

\item  Use the {\bf PRINt} command to view the results of the above 
computation.

\item  After noticing that {\em Learning Statistics With Real Data} is 
the only paperback in the list, you decide to remove that book from the 
data set.

\begin{enumerate}

\item  Since it is always wise to leave the original data set intact, 
use the {\bf COPY} command to copy columns 1 through 5 to columns 6 
through 10.

\item  Now use the {\bf DELEte} command to remove row 11 from columns 
6 through 10.

\item  Use the {\bf INFO} and {\bf PRINt} commands to view the reduced data 
set.

\end{enumerate}

\item  You are now ready to exit Minitab, but before doing so you need to save 
your {\em worksheet}.  

At the MTB $>$ prompt, type 
\begin{center}
{\bf SAVE} 'filename'
\end{center}
to save your {\em worksheet}.  Again, be sure that your filename is 
enclosed in single quotes and contains no periods or other special 
characters, spaces, or file extensions.  The {\bf SAVE} command is the 
second ``critical'' Minitab command.

To exit Minitab, type {\bf STOP} at the MTB $>$ prompt.

\pagebreak

\item  Suppose that you forgot to answer a portion of your assignment.  The 
following exercises will lead you through the process of restarting Minitab 
and opening both your {\em outfile} and your {\em worksheet}.

\begin{enumerate}

\item  At the $>$ prompt, type {\bf minitab}.

\item  At the MTB $>$ prompt, use the {\bf OUTFile} command to open your 
{\em outfile}.

\item  Next use the {\bf RETRieve} command to load your {\em worksheet}.

\item  Use the {\bf INFOrmation} and {\bf PRINt} commands to ensure that you 
have loaded the correct {\em worksheet}.

\item  Since you haven't made any changes to the data, it is not necessary 
to {\bf SAVE} the {\em worksheet} again.  Simply use the {\bf STOP} command 
to exit Minitab.

\end{enumerate}


\item  You will need to use a word processor to complete the assignment.  
The following exercises will lead you through the process of downloading 
the {\em outfile}, opening it in Microsoft Word, editing the word 
document.

\begin{enumerate}

\item  Click on {\bf Window \boldmath $>$ \unboldmath New File Transfer}.

\item  Click once on the name of the {\em outfile}, then
click on {\bf Operation \boldmath $>$ \unboldmath Download}.

\item  Select the folder to which you would like to download the file,
then press {\em Enter}.

\item  Click on {\bf Window \boldmath $>$ \unboldmath Close} to close the
file transfer window.

\item  Type \ {\bf {\tt exit}} \ at the system prompt
\boldmath ($>$), \unboldmath and press {\em Enter}.

\item  Click on {\bf File \boldmath $>$ \unboldmath Exit} to close the SSH
Secure Shell Client.

\item  Open Microsoft Word.

\item  Click on {\bf File \boldmath $>$ \unboldmath Open} to import 
the {\em outfile}.

\item  Select the appropriate folder and click twice on the name of 
the {\em outfile}.  (If the {\em outfile} is not displayed, then 
change the file type to All.)

\item  Select {\bf Plain Text} and click on {\em OK}.

\item  Insert your name, course number, and section number at the top of 
the document, and delete any errors made during the Minitab session.

\end{enumerate}


\end{enumerate}

\end{document}







