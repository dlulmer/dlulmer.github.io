\documentclass{article}
%\setlength{\topmargin}{-1in}
\setlength{\textheight}{9in}
\setlength{\evensidemargin}{-.25in}
\setlength{\oddsidemargin}{-.25in}
\setlength{\textwidth}{6.5in}

\begin{document}
\title{Math 160/263 Minitab Assignment \# 4 - Windows Version} 
\author{Chapter 3 - Sampling}
\date{Worksheet Name - data4.MTW}
\maketitle


\noindent
Of the 1190 tenured professors at the University of Arizona, 61 are aged
31 to 39, 370 are aged 40 to 49, 463 are aged 50 to 59, 271 are aged
60 to 69, and 25 are aged 70 or older.  A stratified random sample of
6 of the professors that are between the ages of 30 and 39, 37 of the
professors that are between the ages of 40 and 49, 46 of the professors
that are between the ages of 50 and 59, 27 of the professors that are
between the ages of 60 and 69, and 3 of the professors that are 70 or
older gives each tenured professor approximately 1 chance in 10 to be
chosen.  This sample design gives every individual in the population
approximately the same chance to be chosen for the sample.  The
professors have been labeled 1 through 1190, and the age of each professor
has been recorded.  This information is given in data4.MTW.

\begin{enumerate}

\item  Use the {\bf Calc \boldmath $>$ \unboldmath Random Data
\boldmath $>$ \unboldmath Sample from Columns} menu
command to select an SRS of size 119 from the 1190 tenured professors.

\item  Use the {\bf Stat \boldmath $>$ \unboldmath Tables \boldmath $>$
\unboldmath Tally} menu command to summarize the results.

\item  How many of the professors in the sample are aged 50 to 59?
How many of the professors in the sample are aged 70 or older?

\item  Explain how the SRS differs from the stratified random sample.

\end{enumerate}

\noindent
Source:  http://daps.arizona.edu.

\end{document}

