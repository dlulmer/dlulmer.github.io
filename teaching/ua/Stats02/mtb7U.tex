\documentclass{article}
%\setlength{\topmargin}{-1in}
\setlength{\textheight}{11in}
\setlength{\evensidemargin}{-.25in}
\setlength{\oddsidemargin}{-.25in}
\setlength{\textwidth}{6.5in}

\begin{document}
\title{Math 160/263 Minitab Assignment \# 7 - Unix Version} 
\author{Chapters 1 and 4 - Examining Distributions and Probability}
\date{}
\maketitle

\begin {enumerate}

\item  The waiting time for the next available station at a particular  
computer lab is uniformly distributed on the interval [0, 15].

\begin{enumerate}

\item  Use the {\bf RANDom} command with the {\bf UNIForm} subcommand 
to simulate the waiting times for 500 students.

\item  Create a histogram of the simulated waiting times, and 
briefly describe the shape of the distribution.

\item  What is the average of the simulated waiting times?
How close is it to 7.5 minutes?

\item  What percent of the simulated waiting times are less 
than 7.5 minutes?  How close is it to 50\%.

\end{enumerate}


\item  The yearly rainfall total for a city in northern California 
is normally distributed with mean 18 inches and standard deviation 
6 inches.

\begin{enumerate}

\item  Use the {\bf CDF} command with the {\bf NORMal} subcommand to 
find the probability that the total rainfall for a randomly selected 
year is less than 10 inches.

\item  Use the {\bf CDF} command with the {\bf NORMal} subcommand to 
find the probability that the total rainfall for a randomly selected 
year is greater than 30 inches.

\item  Use the {\bf CDF} command with the {\bf NORMal} subcommand to 
find the probability that the total rainfall for a randomly selected 
year is between 10 inches and 30 inches.

\item  Use the {\bf INVCdf} command with the {\bf NORMal} subcommand 
to find the third quartile of the distribution of the yearly rainfall 
total for this city.

\end{enumerate}


\end{enumerate}
\end{document}

