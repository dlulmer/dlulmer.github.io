\documentclass{article}
%\setlength{\topmargin}{-1in}
\setlength{\textheight}{12in}
\setlength{\evensidemargin}{-.25in}
\setlength{\oddsidemargin}{-.25in}
\setlength{\textwidth}{6.5in}

\begin{document}
\title{Math 160/263 Minitab Assignment \# 2 - Unix Version} 
\author{Chapter 1 - Describing Distributions}
\date{Worksheet Name - data2.MTW}
\maketitle

\begin{enumerate}

\item  The average annual income in 1999 of workers in the fifty states 
and the District of Columbia are given in data2.MTW.

\begin{enumerate}

\item  Make a {\bf STEMplot} of the data.

\item  Use the {\bf DESCribe} command to find the mean and median of the 
distribution.

\item  What feature of the distribution explains the relationship 
between the mean and median?

\item  Which, if any, observations are suspected outliers?

\item  Approximately what average annual income would place a state in 
the top 25\%?

\item  Approximately what average annual income would place a state in 
the bottom 25\%?

\end{enumerate}

\noindent
Source: \\
{\it 2001 Statistical Abstract of the United States}


\item  The counts of major hurricanes for each year between 1944 and 2000 
are also given in data2.MTW.

\begin{enumerate}

\item  Make a {\bf HISTogram} of the data.

\item  Briefly describe the shape of the distribution.

\item  Make a {\bf TSPLot} of the data.  

\item  Does there appear to be any long term trend in the number of 
major hurricanes per year during the period 1944 to 2000?  Explain.

\end{enumerate}

\noindent
Source: \\
Moore, David S. \& George P. McCabe, {\it Introduction to the Practice 
of Statistics}, 4th ed., W.H. Freeman \& Company, 2002, p. 35.


\end{enumerate}
\end{document}

