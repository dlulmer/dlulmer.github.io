\documentclass{article}
\setlength{\topmargin}{-1.25in}
\setlength{\textheight}{11in}
\setlength{\evensidemargin}{-.25in}
\setlength{\oddsidemargin}{-.25in}
\setlength{\textwidth}{6.5in}

\begin{document}
\title{Math 160/263 Minitab Assignment \# 6 - Unix Version} 
\author{Chapter 4 - Probability}
\date{}
\maketitle

\begin {enumerate}

\item  An urn contains five balls.  One of the balls is numbered with a 
two, one of the balls is numbered with a four, one of the balls is numbered 
with a six, one of the balls is numbered with an eight, and one of the 
balls is numbered with a ten.  Suppose that a ball is selected at random 
from the urn.  The distribution of the number on the ball is given below.

\begin{center}
\begin{tabular}{l|ccccc}

\hline
Number  & 2 & 4 & 6 & 8 & 10 \\
\hline
Prob    & 0.20 & 0.20 & 0.20 & 0.20 & 0.20 \\ 
\hline

\end{tabular}
\end{center} 

\begin{enumerate}

\item  Explain why the distribution is a legitimate discrete probability 
distribution.

\item  Use the {\bf RANDom} command with the {\bf DISCrete} subcommand to 
simulate 100 random selections of a ball from the urn, and use the {\bf TALLy} 
command to summarize the simulated values.  

\item  Produce graphical and numerical descriptions of the simulated values.  
Are the simulated values consistent with the corresponding probability 
distribution?

\item  Use the {\bf LET} command to find the mean and standard deviation of 
the number on a ball selected at random from the urn.  

\item  How close is the mean of the simulated values to the mean of the 
corresponding probability distribution?  How close is the standard deviation 
of the simulated values to the standard deviation of the corresponding 
probability distribution?

\end{enumerate}


\item  Now suppose that two balls are selected at random from the urn 
with replacement.  The distributiion of the average number on the balls 
is given below.

\begin{center}
\begin{tabular}{l|ccccccccc}

\hline
Average & 2 & 3 & 4 & 5 & 6 & 7 & 8 & 9 & 10 \\
\hline
Prob    & 0.04 & 0.08 & 0.12 & 0.16 & 0.20 & 0.16 & 0.12 & 0.08 & 0.04 \\
\hline

\end{tabular}
\end{center} 

\begin{enumerate}

\item  Use the {\bf RANDom} command with the {\bf DISCrete} subcommand to 
simulate two columns of 100 random selections of a ball from the urn, and 
use the {\bf LET} command to find the average of each row of simulated values.

\item  Produce graphical and numerical descriptions of the averages.  
Are the averages consistent with the corresponding probability distribution? 

\item  Use the {\bf LET} command to find the mean and standard deviation of 
the average number on two balls selected at random from the urn with 
replacement.  

\item  How close is the mean of the averages to the mean of the corresponding 
probability distribution?  How close is the standard deviation of the 
averages to the standard deviation of the corresponding probability 
distribution?

\end{enumerate}


\item  A recent Gallup Poll showed that 30\% of Americans believe that 
the U.S. economy is getting better.  Suppose that 20 Americans 
are selected at random and asked for their opinions about the economy.

\begin{enumerate}

\item  Use the {\bf PDF} command with the {\bf BINOmial} subcommand to 
find the probability that all 20 Americans believe that the economy 
is getting better.

\item  Use the {\bf CDF} command with the {\bf BINOmial} subcommand to 
find the probability that more than two of the 20 Americans do not believe 
that the economy is getting better.

\item  Use the {\bf LET} command to find the mean and standard deviation 
of the number of Americans among the 20 that believe that the economy is 
getting better.

\end{enumerate}


\end{enumerate}
\end{document}

