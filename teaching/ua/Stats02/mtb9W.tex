\documentclass{article}
\setlength{\topmargin}{-.25in}
\setlength{\textheight}{11in}
\setlength{\evensidemargin}{-.25in}
\setlength{\oddsidemargin}{-.25in}
\setlength{\textwidth}{6.5in}

\begin{document}
\title{Math 160/263 Minitab Assignment \# 9 - Windows Version} 
\author{Chapter 5 - Introduction to Inference}
\date{Worksheet Name - data9.MTW}
\maketitle

\begin {enumerate}

\item  A manufacturer of pharmaceutical products analyzes a specimen
from each batch of a product to verify the concentration of the active
ingredient.  Because the chemical analysis is not perfectly precise,
repeated measurements on the same specimen give slightly different
results.  The results of 40 repeated measurements are given in
data9.MTW.  These measurements are approximately normally distributed
with standard deviation 0.0068 grams per liter.

\begin{enumerate}

\item  Use the {\bf Stat $>$ Basic Statistics $>$ 1-Sample Z} menu 
command to find 90\%, 95\%, and 99\% confidence intervals for the 
true concentration of the active ingredient.

\item  Explain how the width of the confidence interval changes
as the confidence level increases.

\end{enumerate}


\item  Sulpher compounds cause ``off-odors'' in wine, so oenologists 
(wine experts) have determined the odor threshold, the lowest 
concentration of a compound that the human nose can detect.  For 
example, the odor threshold for dimethyl sulfide (DMS) is given in the 
oenology literature as 25 micrograms per liter of wine ($\mu$g/l).  
Untrained noses may be less sensitive, however.  The DMS odor thresholds 
for 10 beginning students of oenology are given in data9.MTW.  Assume 
that the standard deviation of the odor threshold for untrained noses is 
known to be $\sigma=7 \mu$g/l.

\begin{enumerate}

\item  Create a graph of the data, and briefly describe the shape of
the distribution.

\item  Use the {\bf Stat $>$ Basic Statistics $>$ 1-Sample Z} menu 
command to find a 95\% confidence interval for the 
mean DMS odor threshold among all beginning oenology students.

\end{enumerate}


\item  The Stanford-Binet ``IQ test'' is adjusted so that
the scores for each age group of children are approximately normally
distributed with mean 100 and standard deviation 15.

\begin{enumerate}

\item  If you were to compute a 90\% confidence interval from each of 20
samples from the population of scores on the ``IQ test'', how many would 
you expect to contain the true mean?

\item  Use the {\bf Calc $>$ Random Data $>$ Normal} menu command to 
simulate 50 scores on the ``IQ test'' in each of 20 columns of the 
worksheet.

\item  Use the {\bf Stat $>$ Basic Statistics $>$ 1-Sample Z} menu command
to compute a 90\% confidence interval from each of your columns.

\item  How many of your confidence intervals actually contain the true mean?

\end{enumerate}


\end{enumerate}
\end{document}
